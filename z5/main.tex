\documentclass[fleqn]{article}

\usepackage{polski}
\usepackage[utf8]{inputenc}
\usepackage[polish]{babel}
\usepackage{parskip}
\usepackage{icomma}
\usepackage[a4paper,includeheadfoot,margin=1.27cm]{geometry}
\usepackage{float}
\usepackage{graphicx}
\usepackage{amsmath}
\usepackage[hypcap=true]{subcaption}
\usepackage{xcolor}
\usepackage{transparent}
\usepackage{listings}
\usepackage[colorlinks=true, linkcolor=blue, pdfborder={0 0 0}]{hyperref}

\renewcommand\thesection{\arabic{section}.}
\renewcommand\thesubsection{\alph{subsection})}
\renewcommand\thesubsubsection{}
\newcommand\square[1]{
	\fcolorbox{black}{#1}{\rule{0pt}{6pt}\rule{6pt}{0pt}}
}

\brokenpenalty=1000
\clubpenalty=1000
\widowpenalty=1000

\lstdefinestyle{customc}{
	belowcaptionskip=1\baselineskip,
	breaklines=true,
	frame=L,
	xleftmargin=\parindent,
	language=C,
	showstringspaces=false,
	basicstyle=\footnotesize\ttfamily,
	keywordstyle=\bfseries\color{green!40!black},
	commentstyle=\itshape\color{purple!40!black},
	identifierstyle=\color{blue},
	stringstyle=\color{orange},
}


\title{TM -- Laboratorium 5. \\ \large Ładowanie oraz konwersja tekstu z pliku}
\author{Krystian Chachuła \\ Dawid Gruszczyński \\ Marcin Skrzypkowski}

\begin{document}

\maketitle

\setcounter{page}{0}
\thispagestyle{empty}

\pagebreak

\setcounter{page}{1}

\section{Wstęp}
Na piątym laboratorium mieliśmy za zadanie, realizację ładowania tekstu z pliku do mikrokontrolera MSP430 poprzez port szeregowy RS232 oraz jego konwersję poprzez usuwanie powtórzonych znaków oraz zbędnych spacji, odpowiednią modyfikację wielkich oraz małych liter, a także dostawianie brakujących spacji.


Ostateczny układ składał się z następujących elementów:

\begin{itemize}
	\item \textbf{10\_PS1} (moduł zasilacza)
	\item \textbf{570\_MSP430F14x} (moduł mikrokontrolera Texas Instruments serii MSP430f14x lub F16x)
	\item \textbf{06x\_EIA232\_4} (moduł łącza szeregowego RS232)
\end{itemize}

\section{Implementacja}

\section{Modyfikacje zadania}
W zadaniu dodatkowo wykorzystane zostały dwa kanały sterownika DMA. Jeden do odczytu danych z odbiornika w przypadku ich otrzymania, drugi do wpisywania danych do nadajnika w momencie gdy w buforze znajdują się dane do wysłania oraz nadajnik jest gotowy do transmisji.






\pagebreak
\section{Schemat układu}



\pagebreak
 \section{Program}




\pagebreak
\section{Możliwości udoskonalenia systemu}


\end{document}
\grid
